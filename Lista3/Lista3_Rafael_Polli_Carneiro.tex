% vim: foldmethod=marker: foldmarker=\\beginVimFold,\\endVimFold: fdl=0
\documentclass[11pt,twoside,a4paper]{article}

\usepackage[utf8]{inputenc}
\usepackage[T1]{fontenc}    
\usepackage[brazil,portuguese, english]{babel}

\usepackage{amsmath}
\usepackage{amsfonts}
\usepackage{amssymb}
\usepackage{amsthm}
\usepackage{mathrsfs}

\usepackage[top=2cm, bottom=2cm, left=2cm, right=2cm]{geometry}
\usepackage{mathtools}
\usepackage{hyperref}
\usepackage{enumerate}

\usepackage{bbm} %% \mathbbm{1} gives you the identity function symbol 1

%%% allows you to insert many figures indexed by (a), (b), ... on a figure environment
\usepackage{float} 
\usepackage[caption = false]{subfig}

\usepackage{tikz}
\usetikzlibrary{snakes} %% produces curly arrows on tikz
\usetikzlibrary{matrix} %% for commutative diagrams
\usetikzlibrary{arrows}

\usepackage{ifthen} %% gives if conditionals when using newcommand
%% Garamond fonts 
%% \usepackage{ebgaramond}
%% \usepackage[ugm]{mathdesign}

\title{Álgebra 1 - Prof Nazar\\ Lista 3, Resolução}
\author{Rafael Polli Carneiro, RA: 23201910232}
\date{1o quad 2020} 

%%%%%%%%%%%%%%%%%  Math operators %%%%%%%%%%%%%%%%%%%%%%%
%%% \\beginVimFold
% \DeclareMathOperator {\*}{*}

\DeclareTextFontCommand{\emph}{\bfseries\em} %% redefining \emph{} with bold and italic font
\DeclareMathOperator{\definedAs}{\vcentcolon = }

\DeclareMathOperator {\Imagem}{ Im }

\DeclareMathOperator {\Id}{ \mathbbm{1} }

%%% \\endVimFold
%%%%%%%%%%%%%%%%%%%%%%%%%%%%%%%%%%%%%%%%%%%%%%%%%%%%%%%%%



\setcounter{secnumdepth}{0} %%% this will print the section titles on tableofcontens without inserting numbers.

\begin{document}
\maketitle
\tableofcontents

\section{Exercício 1}
Provemos que todo $R$-módulo livre $L$ é, também, um módulo projetivo. 
Em outras palavras, provemos que, dados os $R$-módulos $L, M, N$, um epimorfismo $f:M \to N$ e um 
homomorfismo $g: L \to N$ então, se $L$ for um módulo livre, existirá um homomorfismo $g': L \to M$ tal
que o diagrama abaixo comuta
\[
    \begin{tikzpicture}
        \matrix[ matrix of math nodes, row sep = 3em, column sep = 3em]
        {
            & |(L)| L \\
            |(M)| M & |(N)| N.\\
        };
        \tikzstyle{every node} = [midway]
        \draw[thick, ->] (L) -- (N) node[right] {$g$};
        \draw[thick, dashed, ->] (L) -- (M) node[left] {$\exists\, g'$};
        \draw[thick, ->] (M) -- (M -| N.west) node[above] {$f$};
    \end{tikzpicture}
\]

Inicialmente, notemos que por $L$ ser um módulo livre então existe uma base $B \subseteq L$ tal que para todo 
$R$-módulo $Q$ e toda função $\phi: B \to Q$ existe um único homomorfismo $\phi'$ tal que o diagrama abaixo comuta,
\[
    \begin{tikzpicture}
        \matrix[ matrix of math nodes, row sep = 3em, column sep = 3em]
        {
            |(B)| B & |(L)| L\\
            & |(Q)| Q,\\
        };
        \tikzstyle{every node} = [midway]
        \draw[thick, dashed, ->] (L) -- (Q) node[right] {$\exists \, \phi'$};
        \draw[thick, ->] (B) -- (Q) node[left] {$\phi$};
        \draw[thick, right hook ->] (B) -- (B -| L.west) node[above] {$i$};
    \end{tikzpicture}
\]
onde $i$ representa  a inclusão.

Consequentemente, por $f: M \to N$ ser um epimorfismo, podemos definir uma função
\begin{align*}
    \phi: B & \to M\\
    x & \mapsto y, \; \text{ tal que } f(y) = g(x),
\end{align*}
que satisfaz $f (\phi(x)) = g(x)$.

Agora, pela propriedade universal dos módulos livres, explicitada  no diagrama acima,
temos garantido a existência de um homomorfismo $\phi': L \to M$ tal que
    \[ \phi' \circ i = \phi. \] 
Finalmente, aplicando o homomorfismo $\phi'$ para todo $x \in L$, o qual podemos escrever como a soma finita
    \[ x = \sum_{a \in B} \lambda_a a,\]
obtemos o desejado:
    \[ \phi'(x) = \sum_{a \in B} \lambda_a \phi(a) \implies f \circ \phi' (x) = \sum_{a \in B} \lambda_a f( \phi(a) ) = \sum_{a \in B} \lambda_a g(a) = g(x). \]

\section{Exercício 2}
Este exercício sai como resultado trivial do Exercício 1 junto com a propriedade de sequências exatas que cindem. De fato,
dado o epimorfismo $f: M \to L$ e a identidade $\Id_L: L \to L$, sabemos, por $L$ ser um módulo livre, que existe um homomorfismo 
$g: L \to M$ tal que
    \[ f \circ g = \Id_L. \]
Finalmente, se considerarmos a sequência
\[
    \begin{tikzpicture}
        \matrix[matrix of math nodes, column sep = 3em]
        {
            |(0)| (0) & |(K)| \ker(f) & |(M)| M & |(L)| L & |(00)| (0),\\
        };
        \tikzstyle{every node} = [midway]
        \draw[thick, ->] (0) -- ( 0 -| K.west);
        \draw[thick, right hook ->] (K) -- ( K -| M.west) node[above] {$i$};
        \draw[thick, ->] (M) -- (M -| L.west) node[above] {$f$};
        \draw[thick, ->] (L) -- (L -| 00.west);
    \end{tikzpicture}   
\]
com $i$ a inclusão, e observarmos que a mesma satisfaz as propriedades
\begin{enumerate}[(i)]
    \item $i: \ker(f) \to M$ é um monomorfismo;
    \item $f$ é um epimorfismo;
    \item existe um homomorfismo $g: L \to M$ tal que $f \circ g = \Id_l$;
\end{enumerate}
ou seja, que ela é uma sequência exata e que cinde, concluímos que
    \[ M \cong \ker(f) \oplus L. \]


\section{Exercício 8}
Seja $P$ um $R$-módulo projetivo finitamente gerado pelo subconjunto $X \subseteq P$, ou seja,
    \[ (X) = P, \]
com
    \[ X = \{m_1, m_2, \ldots, m_n \} \]
e $R$ um anel comutativo com unidade.
Mostremos que  existem $f_1, f_2, \ldots, f_n \in \hom_R(P, R)$ tal que
    \[ x = \sum_{i =1}^n f_i(x) m_i \]
para todo $x \in P$ e, ainda mais,
    \[ f \in \hom_R(P,R) \implies  f = \sum_{i =1}^n f(m_i) f_i. \]

Para provarmos este fato, notemos que a partir do conjunto $X$ podemos induzir o seguinte  $R$-módulo livre
    \[ R^{X} \definedAs \{ (\lambda_x)_{x \in X}; \; \lambda_x \in R, \; \forall x \in X\} \]
com base 
    \[ B = \{ e_x \in R^X; \; x \in X\} \]
onde $e_x = (\delta_{i, x})_{i \in X}$, com $\delta_{i,x}$ o índice de Kronecker.


Agora, note que este módulo livre pode ser escrito como a soma direta
    \[ R^{X} = \bigoplus_{i=1}^n R e_{m_i} \]
o qual, consequentemente, induz as projeções (ortogonais):
\begin{align*}
 \pi_i:  R^{X} &\to \prescript{}{R}R \\
 \sum_{j=1}^n \lambda_j e_{m_j} &\mapsto \lambda_i,
\end{align*}
para todo $i = 1, 2, \ldots, n$.

Finalmente, utilizaremos a hipótese de que $P$ é um módulo projetivo. Antes, observe que podemos definir a função
    \[ \Phi: B \to P,\, e_x \mapsto x, \quad \forall x \in X \]
e, como $R^X$ é um módulo livre, $\Phi$ pode ser estendida de forma única para um homomorfismo,
o qual denotaremos por $\overline{\Phi}$. É imediato que $\overline{\Phi}$ é um epimorfismo pois
    \[ \overline{\Phi}\left( \sum_{x \in X} \alpha_x e_x \right) = \sum_{x \in X} \alpha_x x \]
e $X$ gera $P$. Agora, dada a identidade $\Id_P$ e por $P$ ser um módulo projetivo, existe 
um homomorfismo $g: P \to R^X$ tal que o diagrama abaixo comuta
\[
    \begin{tikzpicture}
        \matrix[ matrix of math nodes, row sep = 3em, column sep = 3em] 
        {
            & |(P)| P \\
            |(RX)| R^X & |(PP)| P,\\
        };
        \tikzstyle{every node} = [midway]
        \draw[thick, ->] (P) -- (PP) node[right] {$\Id_P$};
        \draw[thick, ->, dashed] (P) -- (RX) node[above] {$g$};
        \draw[thick, ->] (RX) -- (RX -| PP.west)  node[above] {$\overline{\Phi}$};
    \end{tikzpicture}
\]
i.e.,
    \[ \overline{\Phi} \circ g = \Id_P. \]
Logo, para todo $x \in P$ temos que 
    \[ g(x) = \sum_{i=1}^n \lambda_i e_{m_i} \]
o qual, aplicando $\overline{\Phi}$ na igualdade acima e usando a comutatividade do diagrama, fornece
    \[ x = \sum_{i=1}^n \lambda_i \overline{\Phi} (e_{m_i}) = \sum_{i=1}^n \pi_i( g(x) )  m_i.\]
Portanto, definindo os homomorfismos 
    \[ f_i \definedAs \pi_i \circ g: P \to \prescript{}{R}R, \; \text{para todo}\;  i = 1, 2, \ldots, n \]
obtemos a primeira igualdade desejada
    \[ x = \sum_{i=1}^n f_i(x)  m_i.\] 
Agora, para obtermos a segunda igualdade, basta, para todo homomorfismo $f:P \to \prescript{}{R}R$, aplicarmos $f$ na
igualdade acima, obtendo
    \[ f(x) =  \sum_{i=1}^n f_i(x)  f(m_i) \]
para todo $x \in P$, ou seja, podemos escrever $f$ como  combinação linear dos $f_1, f_2, \ldots, f_n$:
    \[ f = \sum_{i=1}^n  f(m_i) f_i. \] 
\end{document}

