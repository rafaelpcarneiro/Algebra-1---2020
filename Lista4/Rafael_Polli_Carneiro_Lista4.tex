% vim: foldmethod=marker: foldmarker=\\beginVimFold,\\endVimFold: fdl=0
\documentclass[11pt,twoside,a4paper]{article}

\usepackage[utf8]{inputenc}
\usepackage[T1]{fontenc}    
\usepackage[brazil,portuguese, english]{babel}

\usepackage{amsmath}
\usepackage{amsfonts}
\usepackage{amssymb}
\usepackage{amsthm}
\usepackage{mathrsfs}

\usepackage[top=2cm, bottom=2cm, left=2cm, right=2cm]{geometry}
\usepackage{mathtools}
\usepackage{hyperref}
\usepackage{enumerate}

\usepackage{bbm} %% \mathbbm{1} gives you the identity function symbol 1

%%% allows you to insert many figures indexed by (a), (b), ... on a figure environment
\usepackage{float} 
\usepackage[caption = false]{subfig}

\usepackage{tikz}
\usetikzlibrary{snakes} %% produces curly arrows on tikz
\usetikzlibrary{matrix} %% for commutative diagrams
\usetikzlibrary{arrows}

\usepackage{ifthen} %% gives if conditionals when using newcommand
%% Garamond fonts 
%% \usepackage{ebgaramond}
%% \usepackage[ugm]{mathdesign}

\title{Resolução da Lista 4 - Álgebra I \\ Prof Nazar}
\author{Rafael Polli Carneiro, R.A.: 23201910232}
\date{$1^o$ quad $2020$} 

%%%%%%%%%%%%%%%%%  Math operators %%%%%%%%%%%%%%%%%%%%%%%
%%% \\beginVimFold
% \DeclareMathOperator {\*}{*}

\DeclareTextFontCommand{\emph}{\bfseries\em} %% redefining \emph{} with bold and italic font
\DeclareMathOperator{\definedAs}{\vcentcolon = }

\DeclareMathOperator {\Imagem}{ Im }

\DeclareMathOperator {\Z}{\mathbb{Z}}
\DeclareMathOperator {\N}{\mathbb{N}}
\DeclareMathOperator {\Q}{\mathbb{Q}}

%%% \\endVimFold
%%%%%%%%%%%%%%%%%%%%%%%%%%%%%%%%%%%%%%%%%%%%%%%%%%%%%%%%%


\setcounter{secnumdepth}{0} %%% this will print the section titles on tableofcontens without inserting numbers.

\begin{document}
\maketitle
\tableofcontents

\section{Exercício 3}
Seja o $\Z$-módulo $M = \Z_{(p)} / \Z$ com
    \[ \Z_{(p)}  = \left\{ \frac{a}{p^m}; \; a \in \Z, m \in \N \right\}. \]
Provemos que
\begin{enumerate}[(a)]
    \item $M$ não é noetheriano;
    \item Todo submódulo próprio de $M$ é finito e, portanto, $M$ é artiniano.
\end{enumerate}

O item (a) é fácil de provar, basta tomarmos a sequência de submódulos
    \[ N_i \definedAs \left\{ \frac{a}{p^i} + \Z; \; a \in \Z, m \in \N \right\}, \]
para todo $i \in \N.$ Desta sequência, temos que para todo natural $i$ e para todo inteiro $a$, vale a inclusão
    \[ \frac{a}{p^i} \in N_i \implies \frac{a}{p^i} = \frac{a p}{p^{i+1}} \in N_{i+1}, \]
ou seja, a sequência de submódulos 
\[ N_1\subseteq N_2\subseteq \ldots\subseteq N_i\subseteq N_{i+1}\subseteq \ldots\]
é ascendente e não estacionária, pois, para todo natural $i$,
vale
    \[ \frac{1}{p^{i+1}} \in N_{i+1} \setminus N_i. \]
Portanto, existe uma sequência ascendente e não estacionária de submódulos em $M$. Logo, $M$ não é noetheriano.

Agora, provemos que $M$ é artiniano. Para isto, provemos o item (b), ou seja, que todos os submódulos de $M$ são finitos.
Seja $N$ um submódulo de $M$. Considere o conjunto
    \[ C \definedAs \left\{ \frac{a}{p^m} ; \; a \neq 0, p^m \, \text{são primos entre si}  \text{ e } \frac{a}{p^m} + \Z \in N \right\}. \]
Então, se $C = \emptyset$, temos que
    \[ N = 0 + Z\]
que é finito.
Caso contrário, fixe o elemento
    \[ \frac{a}{p^m} \in C \neq \emptyset. \]
Logo, pelo Teorema de Bezout, existem inteiros $r, s$ tal que
    \[ r a + s p^m = 1.\]
Por esta igualdade somos capazes de inferir que
    \[ \frac{b}{p^i} + \Z \in N, \quad \forall b \in Z, i \in \N \text{ e } i \leq m. \]
De fato, seja $\frac{b}{p^i} + \Z$ um elemento como descrito acima. Logo,
    \[ \frac{b}{p^i} = \frac{b a r}{p^i} + b s p^{m-i} \]
ou seja,
    \[\frac{b}{p^i} + \Z =   \frac{b a r p^{m-i}}{p^m} + \Z = (brp^{m-i}) \left(\frac{a}{p^m} + \Z \right) \in N. \]
Agora, por $\Q$ ser arquimediano, sabemos que para todo $b/p^{n}$, com $b$ um inteiro e $n$ um natural, existe um natural
$n_0$ tal que
    \[ \frac{b}{p^n} \leq n_0 \frac{a}{p^m} \implies \frac{b}{p^n} + \Z \in N. \]
Sendo assim, concluímos que
    \[ N = \left\{ \frac{b}{p^i} + \Z ; \quad \forall b \in \Z, i \in \N \text{ e } i \leq m \right\}. \]
Além do mais, temos que todo $b \in \Z$ pode ser escrito como
    \[ b = \alpha_i p^i + b_i, \]
com $\alpha_i \in \Z$ e $b_i \in [1,2, \ldots, p^i -1] \cap \Z$. Portanto
    \[ N =  \left\{ \frac{b}{p^i} + \Z ; \quad i \leq m \text{ e } b \in [0, 1, \ldots p^i -1] \right\}. \]
Ou seja, $N$ é finito.

Finalmente, como todo submódulo de $M$ é finito temos que que toda sequência descendente de submódulos deve ser estacionária.
De fato, basta observar que a sequência dada pela cardinalidade dos submódulos será decrescente e limitada, portanto convergente.
Desta forma, pela convergência concluímos que a sequência descendente de submódulos é estacionária. Portanto, $M$ é artiniano.

\section{Exercício 6}
Seja $\Q$ o anel dos racionais e $\prescript{}{\Q}\Q$ o $\Q$-módulo dos racionais. Note que todo submódulo $N$ de $\prescript{}{\Q}\Q$ é
da forma
    \[ N = (0) \quad \text{ou} \quad N = \prescript{}{\Q} \Q. \]
Portanto, $\Q$  é uma anel artiniano, já que toda sequência descendente de submódulos é estacionária.

Porém, o subanel $\Z \subseteq \Q$ não é um anel artiniano. Para isto, basta obeservar que a sequência
    \[ 2\Z \supseteq 2^2 \Z \supseteq 2^3 \Z 2^4 \Z \supseteq \cdots \]
é descendente e nunca estacionária, pois
    \[ 2^i \notin 2^{i+1} \Z, \quad \forall i \in \N.\]
Portanto, $\Z$ não será um anel artiniano.

\end{document}

