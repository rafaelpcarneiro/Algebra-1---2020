% vim: foldmethod=marker: foldmarker=\\beginVimFold,\\endVimFold: fdl=0
\documentclass[11pt,twoside,a4paper]{article}

\usepackage[utf8]{inputenc}
\usepackage[T1]{fontenc}    
\usepackage[brazil]{babel}

\usepackage{amsmath}
\usepackage{amsfonts}
\usepackage{amssymb}
\usepackage{amsthm}
\usepackage{mathrsfs}

\usepackage[top=2cm, bottom=2cm, left=2cm, right=2cm]{geometry}
\usepackage{mathtools}
\usepackage{hyperref}
\usepackage{enumerate}

\usepackage{bbm} %% \mathbbm{1} gives you the identity function symbol 1

%%% allows you to insert many figures indexed by (a), (b), ... on a figure environment
\usepackage{float} 
\usepackage[caption = false]{subfig}

\usepackage{tikz}
\usetikzlibrary{snakes} %% produces curly arrows on tikz
\usetikzlibrary{matrix} %% for commutative diagrams
\usetikzlibrary{arrows}

\usepackage{ifthen} %% gives if conditionals when using newcommand
%% Garamond fonts 
%% \usepackage{ebgaramond}
%% \usepackage[ugm]{mathdesign}

\title{Resolução - Lista 2}
\author{Rafael Polli Carneiro}
\date{1o quad - 2020} 

%%%%%%%%%%%%%%%%%  Math operators %%%%%%%%%%%%%%%%%%%%%%%
%%% \\beginVimFold
\DeclareMathOperator {\card}{card}
\DeclareMathOperator {\Id}{\mathbbm{1}}


\newcommand {\ComplexoAbs}[1]{%
    \ifthenelse{ \isempty{#1} }% 
        { (\mathcal{K}, \mathcal{S}) }%
        { (\mathcal{K}_{#1}, \mathcal{S}_{#1}) }%
}


\DeclareTextFontCommand{\emph}{\bfseries\em} %% redefining \emph{} with bold and italic font
\DeclareMathOperator{\definedAs}{\vcentcolon = }

\DeclareMathOperator{\RMod}{ R-\text{módulo}}
\DeclareMathOperator {\Imagem}{ Im }
%%% \\endVimFold
%%%%%%%%%%%%%%%%%%%%%%%%%%%%%%%%%%%%%%%%%%%%%%%%%%%%%%%%%

%%%%%%%%%%%%%%%% Theorems %%%%%%%%%%%%%%%%%%%%%%%%%%%%%%%
%%% \\beginVimFold

\theoremstyle{remark}
\newtheorem{exemplo}{Exemplo}[section]

\theoremstyle{definition}
\newtheorem{observacao}{Observação}[section]
\newtheorem{definicao}{Definição}[section]
\newtheorem*{notacao}{Notação}

\theoremstyle{plain}
\newtheorem{teorema}{Teorema}[section]
\newtheorem{proposicao}{Proposição}[section]
%%% \\endVimFold
%%%%%%%%%%%%%%%%%%%%%%%%%%%%%%%%%%%%%%%%%%%%%%%%%%%%%%%%%


\setcounter{secnumdepth}{0} %%% this will print the section titles on tableofcontens without inserting numbers.

\begin{document}
\maketitle
\tableofcontents

\section{Exercício 5}
Seja $M$ um módulo $\RMod$ simples, i.e., $M \neq (0)$ e seus únicos submódulos são $(0)$ e $M$.
Afirmamos que
\begin{enumerate}[(i)]
    \item Dado $f: M \to N$ um homomorfismo não nulo, então $f$ é um monomorfismo;
    \item Caso, ainda, tivermos na afirmação acima que $N$ é um $\RMod$ simples, então
          $f$ será um isomorfismo;
    \item $\hom_R(M,M)$ é um anel de divisão.
\end{enumerate}

Para provarmos a afirmação (i), basta notarmos que $\ker(f)$ é um submódulo de $M$, que, por sua vez,
por ser um módulo simples implica que
    \[ \ker(f) = (0) \quad \text{ou} \quad \ker(f) = M.\]
Como, por hipótese, $f$ é um homomorfismo não nulo, temos que $\ker(f) = (0)$. Consequentemente,
$f$ é um monomorfismo.

Para  o item (ii), decorre do contra-domínio ser um $\RMod$ simples, e da imagem $\Imagem(f)$ ser um submódulo
do contra domínio, que
    \[ \Imagem(f) = (0) \quad \text{ou} \quad \Imagem(f) = M.\]
Novamente, como $f$ é não  nula, concluímos que $\Imagem(f) = M$ e, portanto, com o item (i) obtemos que
$f$ é um isomorfismo.

Seja o conjunto
    \[\hom_R(M,M) = \{f: M \to M;\; f \, \text{é um homomorfismo} \}. \]
Agora, consideremos a seguinte estrutura algébrica,
    \[ (\hom_R(M,M), +, \circ, \Id), \]
onde, para todo $f,g,h \in \hom_R(M,M)$ e para todo ponto $x \in M$, defini-se
\begin{align*}
    (f + g) (x) = f(x) + g(x);\\
    (f \circ g) (x) = f(g(x)).
\end{align*}
Obviamente,  $(\hom_R(M,M), +)$ é um grupo abeliano, pois $M$ é um grupo abeliano.
Note, também, que
    \[ (h \circ ( f + g) )(x) = h(f(x) + g(x)) = h(f(x)) + h(g(x)) = (h \circ f + h \circ g)(x), \]
i.e., a composição é distributiva à esquerda com a soma. Com um mesmo argumento, provamos
a distributividade à direita. Pela composição de funções ser associativa, e pela identidade ser um homomorfismo,
temos garantidos que
    \[ (\hom_R(M,M) \setminus \{0\}, \circ, \Id) \]
é um monóide. (Aqui, denotamos $0$ como o homomorfismo nulo.) Finalmente, pelo item (ii), temos que
todo elemento de $\hom_R(M,M) \setminus \{0\}$ é um isomorfismo. Consequentemente, dados $x, y \in M$
e $\alpha \in R$, vale, para $f \in\hom_R(M,M) \setminus \{0\}$, que
    \[f^{-1} (x + \alpha y) = f^{-1}( f(\widetilde{x}) + \alpha f(\widetilde{y}) ) \]
com $ f( \widetilde{x} ) = x$ e $ f( \widetilde{y} ) = y$. Portanto,
    \[f^{-1} (x + \alpha y) = f^{-1}( f(\widetilde{x} + \alpha \widetilde{y}) ) = \widetilde{x} + \alpha \widetilde{y}. \]
Ou seja,
    \[f^{-1} (x + \alpha y) = f^{-1}( x )+ \alpha f^{-1}(y). \]
Portanto,
    \[ f \in \hom_R(M,M) \implies f^{-1} \in \hom_R(M,M) \]
e, desta forma, 
    \[ (\hom_R(M,M) \setminus \{0\}, \circ, \Id) \]
é um grupo. Consequentemente, $\hom_R(M,M)$ é um anel de divisão.

\section{Exercício 7}
Seja a sequência exata
\[
    \begin{tikzpicture}
        \matrix [matrix of math nodes, row sep=1cm, column sep=1cm]
        {
            |(M)| M & |(N)| N & |(R)| R & |(S)| S\\
        };
        \tikzstyle{every node} = [midway, above]

        \draw[->] (M) -- (N) node {$f$};
        \draw[->] (N) -- (R) node {$g$};
        \draw[->] (R) -- (S) node {$h$};
    \end{tikzpicture}
\]
Então, as seguintes afirmações são equivalentes:
\begin{enumerate}[(i)]
    \item $f$ é epimorfismo;
    \item $\Imagem(g) = (0)$;
    \item $h$ é monomorfismo.
\end{enumerate}
Provemos as equivalências.

$(i) \implies (ii)$. Se $f$ form um epimorfismo, então vale $\Imagem{f} = N$. Da mesma forma, por estarmos
trabalhando com uma sequência exata, temos que
    \[\ker(g) = \Imagem(f) = N.\]
Logo, temos que $\ker(g) = N \implies g(N) = \{0\}$. Em outras palavras, 
    \[\Imagem(g) = (0). \]

$(ii) \implies (iii).$ Esta implicação é resultado imediato de
    \[ \Imagem(g) = (0) = \ker(h). \]
A igualdade acima fornece que $h$ é um monomorfismo.

$(iii) \implies (i)$. Por $h$ ser um monomorfismo, e por estarmos trabalhando numa sequência exata,
vale
    \[ \ker(h) = (0) = \Imagem(g) \implies \ker(g) = N = \Imagem(f).\]
Portanto, $f$ é um epimorfismo.


\end{document}

